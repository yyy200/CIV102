\documentclass{article}

\usepackage[english]{babel}
\usepackage[utf8]{inputenc}
\usepackage{fancyhdr}
\usepackage{amsmath}
\usepackage{amssymb}
\usepackage{setspace} 
\usepackage{graphicx}
\usepackage{float}
\usepackage{tabularx}


\pagestyle{fancy}
\fancyhf{}
\lhead{CIV102: Formula sheet}
\rhead{Youssef El Mays}
\doublespacing

\newcommand{\SubItem}[1]{
    {\setlength\itemindent{15pt} \item[-] #1}
}

\newcolumntype{L}{>{\centering\arraybackslash}m{3cm}}


\begin{document}
    \title{CIV102: Formula Sheet}

    \section{Stress, Strain}
        \begin{flalign} 
            &\sigma = \frac{F}{A} &\\
            &\epsilon = \frac{\Delta l}{L} &\\
            &\sigma = E\epsilon  &\\
            &\frac{F}{A} = \frac{E \cdot \Delta l}{L}, F = \frac{AE}{L} \cdot \Delta l = k \Delta l  \\
            &k = \frac{AE}{L}  &\\
            &W = \frac{1}{2}k\Delta x^2 
            = \frac{1}{2} F\Delta x 
            = \frac{1}{2} \sigma A \epsilon L 
            = \frac{\sigma^2 V_0}{2E} 
            = \frac{\sigma \epsilon V_0}{2}&\\ 
            &U = \int \sigma d\epsilon &\\
            &W = U \cdot V_0 \rightarrow U = \frac{W}{V_0} &\\
            &\epsilon_{th} = \alpha \Delta T& \\
            &F_{failure} = A\sigma_y& \\
            &F_{allowable} = \frac{A\sigma_y}{FOS} \geq F_{demand} &\\
            &A \geq \frac{FOS \cdot F_{demand}}{\sigma_y}&
        \end{flalign}

    \section{Equilibrium}  
        \begin{flalign}
            &\Sigma F = 0,\;
            \Sigma M = 0,\;if\;a = 0, &\\
            &\;where\;F = ma,\;and\;M = Fd&
        \end{flalign}
    
    \section{Bridges}
        \begin{flalign}
           & T_{supY}=\frac{WL}{2}& \\
           & T_{supX}=\frac{WL^{2}}{8h}& \\
           & T_{supMax}=\sqrt{(\frac{WL}{2})^{2} + (\frac{WL^{2}}{8h})^2},& 
        \end{flalign}
        Given loads placed equidistantly over the span

    \section{Rotation}
        \subsection{Motion}
        \begin{flalign}
           & I_{m} = my^{2},\;for\;a\;single\;point\;mass\\
           & I_{m} = \int_{M} y^2 dm = \rho \int_A y^2 dA &\\
           & I = \int_A y^2 dA&\\
           & I = \frac{bh^3}{12}, \;for\;a\;rectangle\;(comes \;from\;integrating)& \\
           & F=ma=my&\\
           & M=F_{x}y=my^{2} \cdot \alpha = I_{m}\alpha &\\
           & \omega = \frac{d^{2}\theta}{dt^{2}}& \\
           & \theta = \omega_{i}t + \frac{1}{2}\alpha t^{2}&\\   
           & a = \alpha y& \\
           & M = Fy = m\alpha y^2 = I_m \alpha&
        \end{flalign}
        \subsection{Bending}
        \begin{figure}[H]
            \centering
            \includegraphics[width=6cm]{Bending.png}
        \end{figure}
        \begin{flalign}
           & \phi = \frac{d\theta}{dx} &\\
           & L^\prime_{AB} = \phi L_0 \cdot (y +\frac{1}{\phi}) = \phi y L_0 + L_0 &\\
           & \epsilon(y) = \frac{\Delta l}{L_0} = \frac{L^\prime_{AB} - L_0}{L_0} = \phi y,\;\epsilon\; at\; the \; centroidal\; axis\; is\;0&\\
           & \sigma = E\epsilon \rightarrow \sigma(y) = E\phi y &\\
           & \sigma = F/A \rightarrow \Delta F = \sigma(y)\Delta A &\\
           & M =Fd \rightarrow \Delta M = \Delta Fy = \phi E y^2 \Delta A &\\
           & M = \int_A \phi E y^2 dA = \phi E I&
        \end{flalign}

    
    \section{Harmonic Motion}
        \begin{flalign}
            &Pendulum:\;T = 2\pi \sqrt{\frac{l}{g}} &\\
            &Mass\;Spring:\;2\pi \sqrt{\frac{m}{k}} &\\
            &f = \frac{1}{T} &\\
            &F = -kx &\\
            &\omega = \frac{2\pi}{T}& \\
            &a = -\omega^2 x &\\
            &x(t) =A \cdot \sin(\omega t + \phi) &\\
            &v = \omega A \cdot \cos(\omega t + \phi)& \\
            &v = \pm \omega \sqrt{x_0^2 - x^2}& \\
            &\frac{d^2x(t)}{dt^2} = -A\omega^2 \sin(\omega_n t + \phi)& \\
            &E_k = \frac{1}{2}m\omega^2 (x_0^2 - x^2) &\\
            &E_t = \frac{1}{2}m\omega^2 x^2& \\
            &\frac{1}{2}mv^2 = \frac{1}{2}kx^2 \rightarrow v = \sqrt{x\frac{k}{m}}&
        \end{flalign}

        \textbf{Gravity}
        \begin{flalign}
            &m\frac{d^2x(t)}{dt^2} + kx(t)=mg& \\
            &x(t) = A \sin(\omega_nt + \phi) + \Delta_0 &\\
            &k = \frac{mg}{\Delta_0} &\\
            &f_n = \frac{1}{2\pi}\sqrt{\frac{mg}{\Delta_0}\cdot\frac{1}{m}} =\frac{1}{2\pi}\sqrt{\frac{g}{\Delta_0}}&
        \end{flalign}

    \section{Staticaly Determined Structures}
        \begin{table}[H]
            \centering
            \begin{tabularx}{\textwidth}{| X | L | L | c |}
                \hline
                \textbf{Name} & 
                \textbf{Permited Degrees of Freedom} & 
                \textbf{Restrained Degrees of Freedom} &
                \textbf{Reactions} \\
                \hline
                Roller & 
                $\Delta(x \oplus y),\;\theta$ &
                $\Delta y = 0$ &
                $F_y$ \\
                \hline
                Pin &
                $\theta$ &
                $\Delta x = \Delta y = 0$ &
                $F_x,\;F_y$ \\
                \hline 
                Fixed End &
                N/A &
                $\Delta x = \Delta y = \Delta \theta = 0$ &
                $F_x,\;F_y,\;M_{xy}$ \\
                \hline
            \end{tabularx}
        \end{table}

    \section{Truss Brigde}
        \begin{figure}[H]
            \centering
            \includegraphics[width=8cm]{TrussBridge.jpeg}
        \end{figure}
        \begin{enumerate}
            \item Select Geometry
            \item Determine Loads
                \SubItem{Gravity}
                \SubItem{Wind}
            \item Analyze forces in members
            \item Design components
            \item Determine stiffeness
            \item Check dynamic forces 
            \item Iterate
        \end{enumerate}
        \begin{flalign}
            &P_{central\;joint}= W \cdot \frac{s\cdot W_D}{2} &\\
           & P_{end\;joint}= W \cdot \frac{s\cdot W_D}{4} &\\
           & \textbf{Deflection:}\\
            &W_{ext} = \sum^m_{i=1}{\int F_i d\Delta_i}\;m\;Forces& \\
            &W_{int} = \sum^n_{i=1}{\int P_i d\Delta_i}\; n \;members& \\
            &W_{ext} = W_{int} &\\
            &F^\star = virtual\; force &\\
           & \Delta_l = \frac{PL}{AE} &\\
            &F^\star \Delta = \sum{P^\star \Delta_l} = \sum{P^\star  \frac{PL}{AE}}
        \end{flalign}
    
    \section{Buckling}
        \begin{flalign}
           & \phi = \frac{d\theta}{dx} &\\
           & P \cdot y = P \cdot \Delta_{lat} = M&\\
           & M = EI\phi &\\
           & y(x) = A \sin(\omega x + B) &\\
            &\frac{dy}{dx} = A\omega \cos(\omega x +B) &\\
           & \frac{d^2y}{dx^2} = -A\omega^2 \sin(\omega x +B)& \\
            &n\pi =L\cdot \sqrt{\frac{P}{EI}},\;n\;\epsilon \;\mathbb{N}&\\
            &P = \frac{n^2\pi^2 E I}{L^2}& \\
            &P_{crit} = \frac{\pi^2 EI}{L^2} = P_E &\\
            &r = \sqrt{\frac{I}{A}},\;r=\;radius\;of\;gyration&\\
            &\sigma_E = \frac{P_E }{A} = \frac{\pi^2 EI}{AL^2} = \frac{\pi^2 E}{(L/r)^2},\;L/r=\;slenderness\;ratio & \\
            &L/r < 200
        \end{flalign}

    \section{Compression Member}
    \begin{flalign}
        &failure\;envelope = \min\{\sigma_y, \sigma_E\} &\\
        &\sigma_{allowable} = \min\{\frac{1}{FOS_{crush}}\cdot\sigma_y, \frac{1}{FOS_{buckling}}\cdot\sigma_E\} &\\
        &Yield:\;F_{allowable}=\frac{1}{FOS_{crush}}A\sigma_Y \geq F_{demand}& \\
        &A \geq FOS_{crush} \cdot \frac{F_{demand}}{\sigma_Y}&\\
        &Buckling:\;F_{allowable}=\frac{1}{FOS_{buckling}}A\sigma_E \geq F_{demand} &\\
        &I \geq FOS_{buckling} \cdot \frac{F_{demand} L^2}{\pi^2 E}&
    \end{flalign}

    \section{Wind}
    \begin{flalign}
       & W_{wind} = \frac{F_{wind}}{A} = \frac{1}{2}\rho v^2 C_D &\\
        &race\;car:\;C_D=0.2&\\
        &sphere:\;C_D=0.75 &\\
        &boxy\;object:\;C_D=1.5& \\
        &\rho_{air} = 1.2kg/m^3& \\
        &C_D = 1.5 &\\
        &V \geq 170km/h \approx 47.2m/s &\\
        &W_{wind\;ave} = 2.0kPa&\\
        &A_{tributary\;bottom} = \frac{h \cdot s}{2},\; due \; to\; handrail &\\
        &A_{tributary\;top} = \Sigma sh&
    \end{flalign}

    \section{Free vibrations in truss}
    \begin{flalign}
        &f_n = \frac{15.76}{\sqrt{\Delta_0}}\; for\; single\; load& \\
        &f_n = \frac{17.76}{\sqrt{\Delta_0}}\; for\; distributed\; load&\\
        &\textbf{Damping}: \text{Tendency of a system to lose energy as it vibrates} &\\
        &\beta = \text{Damping ratio} \\
        &\frac{m d^2x}{dt^2} + 2\beta\sqrt{mk}\frac{dx}{dt} + kx = mg &\\
        &x(t) = Ae^{-\beta \omega_n t} sin(\omega_n t \sqrt{1 - \beta^2} + \phi) + \Delta_0&
    \end{flalign}

    \section{Forced vibrations}
    \begin{flalign}
        &F(t) = F_0sin(\omega t)+mg &\\
        &\frac{m d^2x}{dt^2} + 2\beta\sqrt{mk}\frac{dx}{dt} + kx = F_0sin(\omega t) mg &\\
        &x(t) = DAF \cdot \frac{F_0}{k}\sin(\omega t + \phi) + \Delta_0,\;\text{At steady state} &\\
        &\textbf{Dynamic Amplification Factor}:\; DAF = \frac{1}{\sqrt{(1-(\frac{f}{f_n})^2)^2+(2\frac{\beta f}{f_n})^2}}&\\
        &\Delta_{max} = \frac{F_{max}}{k}& \\
        &F_{max} = DAF \cdot F_0 + mg &
    \end{flalign}

    \section{Shear forces}
    \begin{flalign}
        &w(x) = \frac{d}{dx}v(x)& \\
        &\Delta V_{12} = \int_{x1}^{x2}w(x) \,dx& \\
        &v(x) = \frac{d}{dx}M(x)& \\
        &\Delta M_{12} = M_2 - M_1 = \int_{x1}^{x2}v(x)\,dx &
    \end{flalign}
\end{document}
